\documentclass[11pt]{article}
\usepackage[letterpaper,margin=1in]{geometry}
\usepackage{mathtools,amssymb,amsfonts,amsthm,empheq,mdframed}
\usepackage{multirow,booktabs,tabularx}
\usepackage{xcolor,listings,realboxes}
\usepackage{courier}
\usepackage{lipsum}
\usepackage{tikz}
\usetikzlibrary{arrows.meta,calc,decorations,shapes.geometric}
%\usepackage{pgfplots}

\definecolor{vb-red}{HTML}{D05353}
\definecolor{vb-blue}{HTML}{464D77}
\definecolor{vb-white}{HTML}{F1ECCE}
\definecolor{vb-sky}{HTML}{9FC2CC}

\lstdefinestyle{mystyle}{
    backgroundcolor=\color{vb-white},   
    commentstyle=\color{green},
    keywordstyle=\color{vb-red},
    stringstyle=\color{vb-blue},
    basicstyle=\footnotesize\ttfamily,
    numberstyle=\ttfamily,
    breaklines=true,       
    captionpos=b, 
    keepspaces=true, 
    numbersep=5pt, 
    tabsize=2,
    numbers=left,
    frame=tBlR,
    showstringspaces=false,
    rulesepcolor=\color{vb-sky}
}

\lstset{style=mystyle}

% https://www.overleaf.com/learn/latex/Code_listing

%%%%%%%%
%%%%%%%%
%%%%%%%%

%TODO: Primitive ranges (how long is a long), Order of operations


\title{Dulles High School CS Club\\Official Reference Sheet}
\date{\today}
\author{Miraj Parikh, Austin Song}

\begin{document}

\lstinputlisting[language=Java]{govikes.java}
{\let\newpage\relax\maketitle} % snippet to maketitle w/out prefixing page

\renewcommand{\arraystretch}{1.25} % it hurts the eyes to keep numbers close together
\section{Primitive Data Types}
\begin{center}\begin{tabular}{llrr}\toprule
  Name            & Initialization              & Range (Java) & Range (C-family) \\\midrule
  Boolean         & \lstinline|bool x|          & True/False & --\\\midrule
  Character & \lstinline|signed char x|   & \(-32768 \leq x < 32768\) & \(-128 \leq x < 128\) \\ 
  Character & \lstinline|unsigned char x|          & \(0 \leq x < 65536\) & \(0 \leq x < 256\) \\
  Byte            & \lstinline|byte x|          & \(-128 \leq x < 128\) & -- \\
                  & \lstinline|unsigned byte x| & \(0 \leq x < 256\) & -- \\
  Short           & \lstinline|short x|         & \(-32768 \leq x < 32768\) & -- \\
                  & \lstinline|unsigned short x|& \(0 \leq x < 65536\) & -- \\
  Integer         & \lstinline|int x|           & \(-2147483648 \leq x < 2147483648\) & -- \\
                  & \lstinline|unsigned int x|  & \(0 \leq x < 4294967295\) & -- \\
\end{tabular}\end{center}

\setlength{\tabcolsep}{1pt}
\renewcommand{\arraystretch}{1}
\section{Order of Operations} 
By descending priority: 
\begin{center}\begin{tabularx}{\textwidth}{clccX}\toprule
  Level               & \quad Name        & Operator          & Direction & \qquad Description \\\midrule
  \multirow{4}{*}{16} & Parentheses       & \lstinline|(x)|   & \multirow{4}{*}{\(\rightarrow\)}& Groups code together \\
                      & Array Access      & \lstinline|x[y]|  & & Access element indexed at \lstinline|int y| of an array \lstinline|x[]|\\
                      & Object Creation   & \lstinline|new|   & & Keyword allowing creation of a new object\\
                      & Member Access     & \lstinline|x.y|   & & Allows member \lstinline|y| to be accessed from parent \lstinline|x| \\ \midrule
  \multirow{2}{*}{15} & U. Post-Increment & \lstinline|x++|   & \multirow{2}{*}{\(\rightarrow\)} & Return the value of \lstinline|int x| before the increment \\
                      & U. Post-Decrement & \lstinline|x--|   & & Returns the value of \lstinline|int x| before the decrement\\\midrule
  \multirow{6}{*}{14} & U. Plus           & \lstinline|+x|    & \multirow{6}{*}{\(\leftarrow\)} & Returns the value of \lstinline|int x|\\
                      & U. Minus          & \lstinline|-x|    & & Returns the negative value of \lstinline|int x| \\
                      & U. L. NOT         & \lstinline|-x|    & & Returns the value of \lstinline|bool x| after flipping it \\
                      & U. B. NOT         & \lstinline|-x|    & & Returns the value of \lstinline|int x| after flipping each bit\\
                      & U. Pre-Increment  & \lstinline|++x|   & & Increments before returning the value of \lstinline|int x| \\
                      & U. Pre-Decrement  & \lstinline|--x|   & & Decrements before returning the value of \lstinline|int x| \\\midrule
                  13  & Cast              & \lstinline|(y) x| & \(\leftarrow\) & Casts variable \lstinline|x| to type \lstinline|y| \\\midrule
                  12  & Multiplicative    & \parbox[c]{1cm}{\centering\lstinline|x * y|\\\lstinline|x / y|\\\lstinline|x \% y|} & \(\rightarrow\) & \\\midrule
                  11  & Additive          & \lstinline|x + y| or \lstinline|x - y| & \(\rightarrow\) & Includes string concatenation \\\midrule
                  10  & Bitshift          & \parbox[c]{1.5cm}{\centering\lstinline|x >> y|\\\lstinline|x << y|\\\lstinline|x >>> y|} & \(\rightarrow\) & Shifts binary expression of \lstinline|int x| by \lstinline|int y| bits\\\midrule
                  9   & Relational        & \parbox[c]{1.5cm}{\centering\lstinline|x < y|\\\lstinline|x <= y|\\\lstinline|x >= y|\\\lstinline|x > y|} & \(\rightarrow\) & \\\midrule
                  8   & Equality          & \parbox[c]{1.5cm}{\centering\lstinline|x = y|\\\lstinline|x != y|} & \(\rightarrow\) & \\\midrule
                  7   & B. AND            & \lstinline|x \& y|    & \(\rightarrow\) & \\\midrule
                  6   & B. XOR            & \lstinline|x \^ y|    & \(\rightarrow\) & \\\midrule
                  5   & B. OR             & \lstinline+x | y+     & \(\rightarrow\) & \\\midrule
                  4   & L. AND            & \lstinline|x \&\& y|  & \(\rightarrow\) & \\\midrule
                  3   & L. OR             & \lstinline+x || y+    & \(\rightarrow\) & \\\midrule
                  2   & Ternary           & \lstinline|x ? y : z| & \(\leftarrow\) & \\\midrule
                  1   & Assignment        & \lstinline!x = y!     & \(\leftarrow\) & Includes all compound assignment operators \\\midrule
                  0   & Lambda / Switch   & \lstinline|x -> y|    & \(\leftarrow\) & \\\bottomrule
\end{tabularx}\end{center}

U. \quad Unary - Operator deals with one single argument

L. \quad Logical - Operator deals with booleans and true/false values

B. \quad Bitwise - Operator manipulates bits of primitive data types

\(J\) - Java-Exclusive \quad \(C\) - C-Family-Exclusive \quad \(P\) - Python-Exclusive

\section{Containers}
\subsection{Data Structures}
Note that any value related to an index, \lstinline|int i|, for these intents and purposes cannot be negative; It would be more appropriate to interpret as \lstinline|unsigned int i|. 
\setlength{\tabcolsep}{5pt}
\begin{center}\begin{tabularx}{\textwidth}{lX}\toprule
  Name & Initialization, Methods, Properties\\
  \multirow{3}{*}{Array} & \lstinline|Type[] x = new Type[int i];| \\
                         & \lstinline|int i -> x.length| \\
                         & \lstinline|x[int j] = Type z;| \\\midrule
  \multirow{5}{*}{\lstinline|ArrayList|} & \lstinline|ArrayList<Type> x = new ArrayList<Type>();| \\
                                         & \lstinline|x.add(Type y)| or \lstinline|x.add(int i, Type y)|\\
                                         & \lstinline|x.remove(int i)| \\
                                         & \lstinline|x.clear()| \\
                                         & \lstinline|boolean x.isEmpty()| \\
  \multirow{4}{*}{\lstinline|HashMap|} & \lstinline|HashMap| \\
\end{tabularx}\end{center}

%\section{shits and giggles}
%\subsection{Alma Mater}
%Dulles High we pledge to you, We’ll be loyal;

%We’ll be true.  May we always bring you honor,

%Glory and acclaim.  As each year goes passing by,

%We will keep your banner high.  Hail to you –

%RED, WHITE, and BLUE Praise your exalted name.

\end{document}
