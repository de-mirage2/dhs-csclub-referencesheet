\documentclass{article}
\usepackage[letterpaper,margin=1in]{geometry}
\usepackage{mathtools,amssymb,amsfonts,amsthm,empheq,mdframed}
\usepackage{multirow,booktabs,tabularx}
\usepackage{xcolor,listings,realboxes}
\usepackage{courier}
\usepackage{lipsum}
\usepackage{tikz}
\usetikzlibrary{arrows.meta,calc,decorations,shapes.geometric}
%\usepackage{pgfplots}

\definecolor{vb-red}{HTML}{D05353}
\definecolor{vb-blue}{HTML}{464D77}
\definecolor{vb-white}{HTML}{F1ECCE}
\definecolor{vb-sky}{HTML}{9FC2CC}

\lstdefinestyle{mystyle}{
    backgroundcolor=\color{vb-white},   
    commentstyle=\color{green},
    keywordstyle=\color{vb-red},
    stringstyle=\color{vb-blue},
    basicstyle=\footnotesize\ttfamily,
    numberstyle=\ttfamily,
    breaklines=true,       
    captionpos=b, 
    keepspaces=true, 
    numbersep=5pt, 
    tabsize=2,
    numbers=left,
    frame=tBlR,
    showstringspaces=false,
    rulesepcolor=\color{vb-sky}
}

\lstset{style=mystyle}

% https://www.overleaf.com/learn/latex/Code_listing

%%%%%%%%

\title{Dulles VikeBytes\\Official Reference Sheet}
\date{\today}
\author{Dulles CS Club}
% Miraj Parikh wrote most of the code with some assistance from Austin Song

\begin{document}

\lstinputlisting[language=Java]{govikes.java}
{\let\newpage\relax\maketitle} % snippet to maketitle w/out prefixing page

\renewcommand{\arraystretch}{1.25} % it hurts the eyes to keep numbers close together
\section{Primitive Data Types (Java)}
\begin{center}\begin{tabular}{lllr}\toprule
  Name      &Size (Bits) &Initialization              &Range \\\midrule
  Boolean   &1           &\lstinline|bool x|          &True/False \\\midrule
  Character &16          &\lstinline|signed char x|   &\(-32768 \leq x < 32768\) \\ 
            &            &\lstinline|unsigned char x| &\(0 \leq x < 65536\) \\
  Byte      &8           &\lstinline|byte x|          &\(-128 \leq x < 128\) \\
            &            &\lstinline|unsigned byte x| &\(0 \leq x < 256\) \\
  Short     &16          &\lstinline|short x|         &\(-32768 \leq x < 32768\) \\
            &            &\lstinline|unsigned short x|&\(0 \leq x < 65536\) \\
  Integer   &32          &\lstinline|int x|           &\(-2147483648 \leq x < 2147483648\) \\
            &            &\lstinline|unsigned int x|  &\(0 \leq x < 4294967295\) \\
  Long      &64          &\lstinline|long x|          &\(-2^{63} \leq x < 2^{63}\) \\
            &            &\lstinline|unsigned long x| &\(0 \leq x < 2^{64}\) \\\midrule
  Float     &32          &\lstinline|float x|         &\(1.5\cdot10^{-45} \leq |x| < 3.4\cdot10^{38}\) \\
            &            &                            &(6-7 significant digits) \\
  Double    &64          &\lstinline|double x|        &\(4.9\cdot10^{-324} \leq |x| < 1.7\cdot10^{308}\) \\
            &            &                            &(15 significant digits) \\
\bottomrule\end{tabular}\end{center}

\setlength{\tabcolsep}{1pt}
\renewcommand{\arraystretch}{1}
\section{Order of Operations} 
Operators are ordered in descending priority (\textit{Lvl.}) and if affiliate, with their respective markings:

U. (Unary) \dotfill Deals with one single primitive data type argument

C. (Conditional) \dotfill Deals with boolean expressions in if statements

BL. (Bitwise-Logical) \dotfill Alters bits of primitives or booleans, depending on passed data

\subsection{Python}
\begin{center}\begin{tabularx}{\textwidth}{clccX}\toprule
  Lvl. & \quad Name & Operator & Direction & Description
\end{tabularx}\end{center}
\subsection{Java}
\begin{center}\begin{tabularx}{\textwidth}{clccX}\toprule
  Lvl.                & \quad Name        & Operator          & Direction & \qquad Description \\\midrule
  \multirow{4}{*}{16} & Parentheses       & \lstinline|(x)|   & \multirow{4}{*}{\(\rightarrow\)}& Groups code together \\
                      & Array Access      & \lstinline|x[y]|  & & Access element indexed at \lstinline|int y| of an array \lstinline|x[]|\\
                      & Object Creation   & \lstinline|new|   & & Keyword allowing creation of a new object\\
                      & Member Access     & \lstinline|x.y|   & & Allows member \lstinline|y| to be accessed from parent \lstinline|x| \\ \midrule
  \multirow{2}{*}{15} & U. Post-Increment & \lstinline|x++|   & \multirow{2}{*}{\(\rightarrow\)} & Return the value of \lstinline|int x| before the increment \\
                      & U. Post-Decrement & \lstinline|x--|   & & Returns the value of \lstinline|int x| before the decrement\\\midrule
  \multirow{6}{*}{14} & U. Plus           & \lstinline|+x|    & \multirow{6}{*}{\(\leftarrow\)} & Returns the value of \lstinline|int x|\\
                      & U. Minus          & \lstinline|-x|    & & Returns the negative value of \lstinline|int x| \\
                      & U. BL. NOT         & \lstinline|-x|    & & Returns the value of \lstinline|int x| after flipping each bit\\
                      & U. C. NOT         & \lstinline|-x|    & & Returns the value of \lstinline|bool x| after flipping it \\
                      & U. Pre-Increment  & \lstinline|++x|   & & Increments before returning the value of \lstinline|int x| \\
                      & U. Pre-Decrement  & \lstinline|--x|   & & Decrements before returning the value of \lstinline|int x| \\\midrule
                  13  & Cast              & \lstinline|(y) x| & \(\leftarrow\) & Casts variable \lstinline|x| to type \lstinline|y| \\\midrule
                  12  & Multiplicative    & \parbox[c]{1cm}{\centering\lstinline|x * y|\\\lstinline|x / y|\\\lstinline|x \% y|} & \(\rightarrow\) & \\\midrule
                  11  & Additive          & \lstinline|x + y| or \lstinline|x - y| & \(\rightarrow\) & Includes string concatenation \\\midrule
                  10  & Bitshift          & \parbox[c]{1.5cm}{\centering\lstinline|x >> y|\\\lstinline|x << y|\\\lstinline|x >>> y|} & \(\rightarrow\) & Shifts binary expression of \lstinline|int x| by \lstinline|int y| bits\\\midrule
                  9   & Relational        & \parbox[c]{1.5cm}{\centering\lstinline|x < y|\\\lstinline|x <= y|\\\lstinline|x >= y|\\\lstinline|x > y|} & \(\rightarrow\) & \\\midrule
                  8   & Equality          & \parbox[c]{1.5cm}{\centering\lstinline|x = y|\\\lstinline|x != y|} & \(\rightarrow\) & \\\midrule
                  7   & BL. AND           & \lstinline|x \& y|    & \(\rightarrow\) & \\\midrule
                  6   & BL. XOR           & \lstinline|x \^ y|    & \(\rightarrow\) & \\\midrule
                  5   & BL. OR            & \lstinline+x | y+     & \(\rightarrow\) & \\\midrule
                  4   & C. AND            & \lstinline|x \&\& y|  & \(\rightarrow\) & \\\midrule
                  3   & C. OR             & \lstinline+x || y+    & \(\rightarrow\) & \\\midrule
                  2   & Ternary           & \lstinline|x ? y : z| & \(\leftarrow\) & \\\midrule
                  1   & Assignment        & \lstinline!x = y!     & \(\leftarrow\) & Includes all compound assignment operators \\\midrule
                  0   & Lambda / Switch   & \lstinline|x -> y|    & \(\leftarrow\) & \\\bottomrule
\end{tabularx}\end{center}

\section{Containers}
\subsection{Data Structures}
Let \lstinline|T| and \lstinline|U| refer to the type of an object (i.e., \lstinline|float| or \lstinline|Integer|). Note that for data structures outside of Arrays, non-primitive types must be used.
Note that any value related to index/sizes (\lstinline|int i|, \lstinline|int j|, or \lstinline|int n|), for these intents, cannot be negative since negative indices do not exist in Java nor C++.
\setlength{\tabcolsep}{3pt}
\begin{center}\begin{tabularx}{\textwidth}{llX}\toprule
  Name & Code & Description\\
  \multirow{3}{*}{Array} & \lstinline|T[] x = new T[n];| & Initialize an array \(x\) of size \(n\)\\
                         & \lstinline|x[i] = y;| & Set element of \(x\) at index \(i\) to \(y\)\\
                         & \lstinline|n = x.length;| & Record the length of array \(x\) \\\midrule
  \multirow{7}{*}[-0.5em]{\lstinline|ArrayList|} & \lstinline|List<T> x = new ArrayList<T>();| & Initialize an ArrayList \(x\) of type \(T\)\\
                                         & \lstinline|x.add(y);| or \lstinline|x.add(i, y);| & Insert element \(y\) to \(x\) at index \(i\); if \(i\) is not given, then insert at the end of \(x\)\\
                                         & \lstinline|x.get(y)| & Return the element located at index \(i\) in \(x\) \\
                                         & \lstinline|x.remove(i);| & Return the element located at index \(i\) in \(x\), then delete it\\
                                         & \lstinline|x.clear();| & Remove all elements within \(x\)\\
                                         & \lstinline|x.isEmpty()| & Return a boolean indicating whether \(x\) is empty or not\\
                                         & \lstinline|x.size()| & Return the count of elements within \(x\)\\\midrule
  \multirow{8}{*}[-2em]{\lstinline|HashMap|} & \lstinline|Map<T, U> x = new HashMap<>();| & Initialize a HashMap \(x\) mapping \(T\) objects to \(U\) objects\\
                                       & \lstinline|x.put(y, z);| & Write entry \(y-z\) to \(x\)\\
                                       & \lstinline|x.get(y);| & Return the value of the entry in \(x\) with a unique key \(y\)\\
                                       & \lstinline|x.getOrDefault(y, z);| & Return the value of the entry in \(x\) with a unique key \(y\); if there is no such entry, then return \(z\) by default.\\
                                       & \lstinline|x.containsKey(y);| & Return a boolean indicating whether \(x\) contains a entry that has a key \(y\)\\
                                       & \lstinline|x.containsValue(y);| & Return a boolean indicating whether \(x\) contains a entry that has a value \(y\)\\
                                       & \lstinline|x.remove(y);| & Return the value of the entry in \(x\) with a unique key \(y\), then delete it\\
                                       & \lstinline|x.size();| & Return the count of entries within \(x\)
\\\bottomrule\end{tabularx}\end{center}

%\section{trolling}
%\subsection{Alma Mater}
%Dulles High we pledge to you, We’ll be loyal;

%We’ll be true.  May we always bring you honor,

%Glory and acclaim.  As each year goes passing by,

%We will keep your banner high.  Hail to you –

%RED, WHITE, and BLUE Praise your exalted name.

\end{document}
